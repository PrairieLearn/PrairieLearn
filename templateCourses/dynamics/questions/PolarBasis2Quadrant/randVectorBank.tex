\documentclass{article}
\usepackage{booktabs}
\usepackage{caption}
\captionsetup[table]{position=top}
\usepackage{amsmath,amssymb}
\usepackage{graphicx}
\usepackage[margin=2.5cm]{geometry}
\usepackage{hyperref}
\usepackage{tikz}

\begin{document}

\setlength{\parindent}{0pt}
\setlength{\parskip}{0.6em}

\vspace{1em}
\begin{center}
  \Large \bf Spring 2013 - Midterm 1 - Q31\\[0.5em]
\end{center}

%
% Reference: http://tex.stackexchange.com/questions/100509/equations-inside-tabular-environment-no-vertical-spacing
%
\begin{table}[htb!]
\centering
\caption{The bank of unit vectors}
\label{tbl:bank}
$\begin{array}{ccc}
\toprule
\mathbf{\theta} & 
\multicolumn{1}{c}{Polar Unit Vectors} & 
\multicolumn{1}{c}{Quadrant} \\
%
\midrule
\addlinespace
%
\mathbf{30^{\circ}} & 
  \begin{aligned}
    \hat{e}_r      &=  \frac{\sqrt{3}}{2} \hat\imath + \frac{1}{2}        \hat\jmath \\[1ex]
    \hat{e}_\theta &= -\frac{1}{2}        \hat\imath + \frac{\sqrt{3}}{2} \hat\jmath
  \end{aligned} &
%
  \begin{aligned}
    1 \\[1ex]
    1
  \end{aligned} \\
%
\addlinespace
\midrule
\addlinespace
%
\mathbf{45^{\circ}} & 
  \begin{aligned}
    \hat{e}_r      &=  \frac{1}{\sqrt{2}} \hat\imath + \frac{1}{\sqrt{2}} \hat\jmath \\[1ex]
    \hat{e}_\theta &= -\frac{1}{\sqrt{2}} \hat\imath + \frac{1}{\sqrt{2}} \hat\jmath
  \end{aligned} &
%
  \begin{aligned}
    1 \\[1ex]
    1
  \end{aligned} \\
%
\addlinespace
\midrule
\addlinespace
%
\mathbf{60^{\circ}} & 
  \begin{aligned}
    \hat{e}_r      &=  \frac{1}{2}        \hat\imath + \frac{\sqrt{3}}{2} \hat\jmath \\[1ex]
    \hat{e}_\theta &= -\frac{\sqrt{3}}{2} \hat\imath + \frac{1}{2}        \hat\jmath
  \end{aligned} &
%
  \begin{aligned}
    1 \\[1ex]
    1
  \end{aligned} \\
%
\addlinespace
\midrule
\addlinespace
\midrule
\addlinespace
%
% ------------------- Quadrant 2 -------------------
%
\mathbf{120^{\circ}} & 
  \begin{aligned}
    \hat{e}_r      &= -\frac{1}{2}        \hat\imath + \frac{\sqrt{3}}{2} \hat\jmath\\[1ex]
    \hat{e}_\theta &= -\frac{\sqrt{3}}{2} \hat\imath - \frac{1}{2}        \hat\jmath
  \end{aligned} &
%
  \begin{aligned}
    2 \\[1ex]
    2
  \end{aligned} \\
%
\addlinespace
\midrule
\addlinespace
%
\mathbf{135^{\circ}} & 
  \begin{aligned}
    \hat{e}_r      &=  \frac{1}{\sqrt{2}} \hat\imath + \frac{1}{\sqrt{2}} \hat\jmath \\[1ex]
    \hat{e}_\theta &= -\frac{1}{\sqrt{2}} \hat\imath - \frac{1}{\sqrt{2}} \hat\jmath
  \end{aligned} &
%
  \begin{aligned}
    2 \\[1ex]
    2
  \end{aligned} \\
%
\addlinespace
\midrule
\addlinespace
%
\mathbf{150^{\circ}} & 
  \begin{aligned}
    \hat{e}_r      &= -\frac{\sqrt{3}}{2} \hat\imath + \frac{1}{2}        \hat\jmath \\[1ex]
    \hat{e}_\theta &= -\frac{1}{2}        \hat\imath - \frac{\sqrt{3}}{2} \hat\jmath 
  \end{aligned} &
%
  \begin{aligned}
    2 \\[1ex]
    2
  \end{aligned} \\
%
\addlinespace
\bottomrule
%
\end{array}$
%
\end{table}
%
%
% ------------------- Quadrant 3 -------------------
%
\begin{table}[htb!]
\centering
\caption{Continued: The bank of unit vectors}
\label{tbl:bank_cont}
$\begin{array}{ccc}
\toprule
\mathbf{\theta} & 
\multicolumn{1}{c}{Polar Unit Vectors} & 
\multicolumn{1}{c}{Quadrant} \\
%
\midrule
\addlinespace
%
\mathbf{210^{\circ}} & 
  \begin{aligned}
    \hat{e}_r      &= -\frac{\sqrt{3}}{2} \hat\imath - \frac{1}{2}        \hat\jmath \\[1ex]
    \hat{e}_\theta &=  \frac{1}{2}        \hat\imath - \frac{\sqrt{3}}{2} \hat\jmath
  \end{aligned} &
%
  \begin{aligned}
    3 \\[1ex]
    3
  \end{aligned} \\
%
\addlinespace
\midrule
\addlinespace
%
\mathbf{225^{\circ}} & 
  \begin{aligned}
    \hat{e}_r      &= -\frac{1}{\sqrt{2}} \hat\imath - \frac{1}{\sqrt{2}} \hat\jmath \\[1ex]
    \hat{e}_\theta &=  \frac{1}{\sqrt{2}} \hat\imath - \frac{1}{\sqrt{2}} \hat\jmath
  \end{aligned} &
%
  \begin{aligned}
    3 \\[1ex]
    3
  \end{aligned} \\
%
\addlinespace
\midrule
\addlinespace
%
\mathbf{240^{\circ}} & 
  \begin{aligned}
    \hat{e}_r      &= -\frac{1}{2}        \hat\imath - \frac{\sqrt{3}}{2} \hat\jmath \\[1ex]
    \hat{e}_\theta &=  \frac{\sqrt{3}}{2} \hat\imath - \frac{1}{2}        \hat\jmath
  \end{aligned} &
%
  \begin{aligned}
    3 \\[1ex]
    3
  \end{aligned} \\
%
\addlinespace
\midrule
\addlinespace
\midrule
\addlinespace
%
% ------------------- Quadrant 4 -------------------
%
\mathbf{300^{\circ}} & 
  \begin{aligned}
    \hat{e}_r      &=  \frac{1}{2}        \hat\imath - \frac{\sqrt{3}}{2} \hat\jmath \\[1ex]
    \hat{e}_\theta &=  \frac{\sqrt{3}}{2} \hat\imath + \frac{1}{2}        \hat\jmath
  \end{aligned} &
%
  \begin{aligned}
    4 \\[1ex]
    4
  \end{aligned} \\
%
\addlinespace
\midrule
\addlinespace
%
\mathbf{315^{\circ}} & 
  \begin{aligned}
    \hat{e}_r      &=  \frac{1}{\sqrt{2}} \hat\imath - \frac{1}{\sqrt{2}} \hat\jmath \\[1ex]
    \hat{e}_\theta &=  \frac{1}{\sqrt{2}} \hat\imath + \frac{1}{\sqrt{2}} \hat\jmath
  \end{aligned} &
%
  \begin{aligned}
    4 \\[1ex]
    4
  \end{aligned} \\
%
\addlinespace
\midrule
\addlinespace
%
\mathbf{330^{\circ}} & 
  \begin{aligned}
    \hat{e}_r      &=  \frac{\sqrt{3}}{2} \hat\imath - \frac{1}{2}        \hat\jmath \\[1ex]
    \hat{e}_\theta &=  \frac{1}{2}        \hat\imath + \frac{\sqrt{3}}{2} \hat\jmath
  \end{aligned} &
%
  \begin{aligned}
    4 \\[1ex]
    4
  \end{aligned} \\
%
% --------------------------------------------------------
%
\addlinespace
\bottomrule
%
\end{array}$
%
\end{table}


\end{document}
